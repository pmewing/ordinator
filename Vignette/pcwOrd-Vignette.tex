% Options for packages loaded elsewhere
\PassOptionsToPackage{unicode}{hyperref}
\PassOptionsToPackage{hyphens}{url}
%
\documentclass[
]{article}
\usepackage{lmodern}
\usepackage{amssymb,amsmath}
\usepackage{ifxetex,ifluatex}
\ifnum 0\ifxetex 1\fi\ifluatex 1\fi=0 % if pdftex
  \usepackage[T1]{fontenc}
  \usepackage[utf8]{inputenc}
  \usepackage{textcomp} % provide euro and other symbols
\else % if luatex or xetex
  \usepackage{unicode-math}
  \defaultfontfeatures{Scale=MatchLowercase}
  \defaultfontfeatures[\rmfamily]{Ligatures=TeX,Scale=1}
\fi
% Use upquote if available, for straight quotes in verbatim environments
\IfFileExists{upquote.sty}{\usepackage{upquote}}{}
\IfFileExists{microtype.sty}{% use microtype if available
  \usepackage[]{microtype}
  \UseMicrotypeSet[protrusion]{basicmath} % disable protrusion for tt fonts
}{}
\makeatletter
\@ifundefined{KOMAClassName}{% if non-KOMA class
  \IfFileExists{parskip.sty}{%
    \usepackage{parskip}
  }{% else
    \setlength{\parindent}{0pt}
    \setlength{\parskip}{6pt plus 2pt minus 1pt}}
}{% if KOMA class
  \KOMAoptions{parskip=half}}
\makeatother
\usepackage{xcolor}
\IfFileExists{xurl.sty}{\usepackage{xurl}}{} % add URL line breaks if available
\IfFileExists{bookmark.sty}{\usepackage{bookmark}}{\usepackage{hyperref}}
\hypersetup{
  pdftitle={pcwOrd: (partial) (constrained) (weighted) ordination},
  pdfauthor={Patrick Ewing},
  hidelinks,
  pdfcreator={LaTeX via pandoc}}
\urlstyle{same} % disable monospaced font for URLs
\usepackage[margin=1in]{geometry}
\usepackage{color}
\usepackage{fancyvrb}
\newcommand{\VerbBar}{|}
\newcommand{\VERB}{\Verb[commandchars=\\\{\}]}
\DefineVerbatimEnvironment{Highlighting}{Verbatim}{commandchars=\\\{\}}
% Add ',fontsize=\small' for more characters per line
\usepackage{framed}
\definecolor{shadecolor}{RGB}{248,248,248}
\newenvironment{Shaded}{\begin{snugshade}}{\end{snugshade}}
\newcommand{\AlertTok}[1]{\textcolor[rgb]{0.94,0.16,0.16}{#1}}
\newcommand{\AnnotationTok}[1]{\textcolor[rgb]{0.56,0.35,0.01}{\textbf{\textit{#1}}}}
\newcommand{\AttributeTok}[1]{\textcolor[rgb]{0.77,0.63,0.00}{#1}}
\newcommand{\BaseNTok}[1]{\textcolor[rgb]{0.00,0.00,0.81}{#1}}
\newcommand{\BuiltInTok}[1]{#1}
\newcommand{\CharTok}[1]{\textcolor[rgb]{0.31,0.60,0.02}{#1}}
\newcommand{\CommentTok}[1]{\textcolor[rgb]{0.56,0.35,0.01}{\textit{#1}}}
\newcommand{\CommentVarTok}[1]{\textcolor[rgb]{0.56,0.35,0.01}{\textbf{\textit{#1}}}}
\newcommand{\ConstantTok}[1]{\textcolor[rgb]{0.00,0.00,0.00}{#1}}
\newcommand{\ControlFlowTok}[1]{\textcolor[rgb]{0.13,0.29,0.53}{\textbf{#1}}}
\newcommand{\DataTypeTok}[1]{\textcolor[rgb]{0.13,0.29,0.53}{#1}}
\newcommand{\DecValTok}[1]{\textcolor[rgb]{0.00,0.00,0.81}{#1}}
\newcommand{\DocumentationTok}[1]{\textcolor[rgb]{0.56,0.35,0.01}{\textbf{\textit{#1}}}}
\newcommand{\ErrorTok}[1]{\textcolor[rgb]{0.64,0.00,0.00}{\textbf{#1}}}
\newcommand{\ExtensionTok}[1]{#1}
\newcommand{\FloatTok}[1]{\textcolor[rgb]{0.00,0.00,0.81}{#1}}
\newcommand{\FunctionTok}[1]{\textcolor[rgb]{0.00,0.00,0.00}{#1}}
\newcommand{\ImportTok}[1]{#1}
\newcommand{\InformationTok}[1]{\textcolor[rgb]{0.56,0.35,0.01}{\textbf{\textit{#1}}}}
\newcommand{\KeywordTok}[1]{\textcolor[rgb]{0.13,0.29,0.53}{\textbf{#1}}}
\newcommand{\NormalTok}[1]{#1}
\newcommand{\OperatorTok}[1]{\textcolor[rgb]{0.81,0.36,0.00}{\textbf{#1}}}
\newcommand{\OtherTok}[1]{\textcolor[rgb]{0.56,0.35,0.01}{#1}}
\newcommand{\PreprocessorTok}[1]{\textcolor[rgb]{0.56,0.35,0.01}{\textit{#1}}}
\newcommand{\RegionMarkerTok}[1]{#1}
\newcommand{\SpecialCharTok}[1]{\textcolor[rgb]{0.00,0.00,0.00}{#1}}
\newcommand{\SpecialStringTok}[1]{\textcolor[rgb]{0.31,0.60,0.02}{#1}}
\newcommand{\StringTok}[1]{\textcolor[rgb]{0.31,0.60,0.02}{#1}}
\newcommand{\VariableTok}[1]{\textcolor[rgb]{0.00,0.00,0.00}{#1}}
\newcommand{\VerbatimStringTok}[1]{\textcolor[rgb]{0.31,0.60,0.02}{#1}}
\newcommand{\WarningTok}[1]{\textcolor[rgb]{0.56,0.35,0.01}{\textbf{\textit{#1}}}}
\usepackage{graphicx,grffile}
\makeatletter
\def\maxwidth{\ifdim\Gin@nat@width>\linewidth\linewidth\else\Gin@nat@width\fi}
\def\maxheight{\ifdim\Gin@nat@height>\textheight\textheight\else\Gin@nat@height\fi}
\makeatother
% Scale images if necessary, so that they will not overflow the page
% margins by default, and it is still possible to overwrite the defaults
% using explicit options in \includegraphics[width, height, ...]{}
\setkeys{Gin}{width=\maxwidth,height=\maxheight,keepaspectratio}
% Set default figure placement to htbp
\makeatletter
\def\fps@figure{htbp}
\makeatother
\setlength{\emergencystretch}{3em} % prevent overfull lines
\providecommand{\tightlist}{%
  \setlength{\itemsep}{0pt}\setlength{\parskip}{0pt}}
\setcounter{secnumdepth}{5}
% https://github.com/rstudio/rmarkdown/issues/337
\let\rmarkdownfootnote\footnote%
\def\footnote{\protect\rmarkdownfootnote}

% https://github.com/rstudio/rmarkdown/pull/252
\usepackage{titling}
\setlength{\droptitle}{-2em}

\pretitle{\vspace{\droptitle}\centering\huge}
\posttitle{\par}

\preauthor{\centering\large\emph}
\postauthor{\par}

\predate{\centering\large\emph}
\postdate{\par}

\title{pcwOrd: (partial) (constrained) (weighted) ordination}
\author{Patrick Ewing}
\date{2020-03-04}

\begin{document}
\maketitle

\tableofcontents
\newpage

\hypertarget{motivation}{%
\section{Motivation}\label{motivation}}

Many ordination packages exist for ecologists using R - \texttt{vegan}
is excellent, for example. But none readily perform (to my knowledge)
weighted, partial constrained ordinations.

The motivation for this is to analyze compositional ecological data -
especially high throughput sequencing data - with methods that are
robust, reproducible, and transparent. Compositional data contains only
relative data (think relative abundance), and so requires a bit of extra
care to analyze. The dominant approach is to perform a log-ratio
transformation, then use principle components analysis - combined, this
is called log-ratio analysis. For an enterance into this literature, see
Greenacre and Aitchison (2002), Greenacre and Lewi (2009), and Quinn et
al (2018).

In particular, Greenacre and Lewi (2002) suggest using weighted
log-ratio analysis, as unweighted log-ratio analysis is susceptible to
noise in low-abundance features. Raw values of low-abundance features
have higher relative variance than raw values of high-abundance
features; therefore, the log-ratios of low abundance features are also
noisy. When low-abundance features dominate a solution the distances and
inferences made about samples is less robust to error and noise. Note
that correspondence analysis also is suceptible to dominance by
low-abundance features.

Weighted log-ratio analysis down-weights these uncertain, low-abundance
features. As a result, the observed distances among samples should be
more reproducible across experiments.

\texttt{vegan} can perform constrained and partiald log-ratio analysis
if the community matrix is log-transformed beforehand. However, it does
not provide a straightforward way to weight columns and working with
ordination objects is not intuitive. \texttt{easyCODA} allows weighted
and constrained log-ratio analysis, but does not allow partialing of
nuisance effects and does not provide hypothesis testing.
\texttt{pcwOrd} allows partialed, constrained, and weighted ordinations,
and also provides utility functions for investigating ordination
objects, visualization, and hypothesis testing. The code is written to
be self-documenting. Maybe you'll agree.

This isn't an R package yet, so load the functions with
\texttt{source()}:

\begin{Shaded}
\begin{Highlighting}[]
\NormalTok{libs =}\StringTok{ }\KeywordTok{c}\NormalTok{(}\StringTok{'easyCODA'}\NormalTok{,  }\CommentTok{# log-ratio analysis and CLR analysis}
         \StringTok{'vegan'}\NormalTok{)     }\CommentTok{# multivariate statistics}
\ControlFlowTok{for}\NormalTok{(i }\ControlFlowTok{in}\NormalTok{ libs) \{}
  \ControlFlowTok{if}\NormalTok{ (}\OperatorTok{!}\KeywordTok{require}\NormalTok{(i, }\DataTypeTok{character.only=}\OtherTok{TRUE}\NormalTok{)) \{}
    \KeywordTok{install.packages}\NormalTok{(i)}
    \ControlFlowTok{if}\NormalTok{ (}\OperatorTok{!}\KeywordTok{require}\NormalTok{(i, }\DataTypeTok{character.only=}\OtherTok{TRUE}\NormalTok{)) }\KeywordTok{stop}\NormalTok{(}\KeywordTok{paste}\NormalTok{(}\StringTok{"Cannot load"}\NormalTok{, i))}
\NormalTok{  \}\}}

\CommentTok{# pcwOrd}
\KeywordTok{load}\NormalTok{(}\StringTok{'pcwOrd_0.1.Rdata'}\NormalTok{)}

\CommentTok{# data}
\NormalTok{spider =}\StringTok{ }\KeywordTok{readRDS}\NormalTok{(}\StringTok{'Spider.RDS'}\NormalTok{)}
\end{Highlighting}
\end{Shaded}

This vignette will demonstrate the features of pcwOrd: ordination,
hypothesis-testing, and visualization. I'll use the \texttt{spiders}
dataset from \texttt{mvabund}. This dataset has two tables:

\begin{enumerate}
\def\labelenumi{\arabic{enumi}.}
\tightlist
\item
  A \texttt{data.frame} of environmental variables, to which I'll add a
  categorical version of \texttt{soil.dry}*.
\item
  A \texttt{matrix} of community data.
\end{enumerate}

Certain pcwOrd functions use the rownames of the environmental and
community data to cross-reference, so we want to ensure both tables have
rownames. Usually these are something meaningful, like sample IDs.

*technically this should be ordinal, but we'll treat it as categorical.

\begin{Shaded}
\begin{Highlighting}[]
\CommentTok{# environmental data}
\NormalTok{envi =}\StringTok{ }\KeywordTok{as.data.frame}\NormalTok{(spider}\OperatorTok{$}\NormalTok{x)}

\CommentTok{# categorical soil wetness}
\NormalTok{cats =}\StringTok{ }\KeywordTok{c}\NormalTok{(}\StringTok{'dry'}\NormalTok{, }\StringTok{'damp'}\NormalTok{, }\StringTok{'wet'}\NormalTok{, }\StringTok{'soaking'}\NormalTok{)}
\NormalTok{cc =}\StringTok{ }\KeywordTok{ceiling}\NormalTok{(envi}\OperatorTok{$}\NormalTok{soil.dry)}
\NormalTok{envi}\OperatorTok{$}\NormalTok{soil.cat =}\StringTok{ }\KeywordTok{factor}\NormalTok{(cats[cc],}
                       \DataTypeTok{levels=}\NormalTok{cats)}

\CommentTok{# community data}
\NormalTok{comm =}\StringTok{ }\KeywordTok{as.matrix}\NormalTok{(spider}\OperatorTok{$}\NormalTok{abund)}

\CommentTok{# rownames}
\NormalTok{nobs =}\StringTok{ }\KeywordTok{nrow}\NormalTok{(comm)}
\NormalTok{row_names =}\StringTok{ }\KeywordTok{c}\NormalTok{(letters[], LETTERS[])[}\DecValTok{1}\OperatorTok{:}\NormalTok{nobs]}
\KeywordTok{rownames}\NormalTok{(envi) =}\StringTok{ }\NormalTok{row_names}
\KeywordTok{rownames}\NormalTok{(comm) =}\StringTok{ }\NormalTok{row_names}

\KeywordTok{str}\NormalTok{(envi)}
\end{Highlighting}
\end{Shaded}

\begin{verbatim}
## 'data.frame':    28 obs. of  7 variables:
##  $ soil.dry     : num  2.33 3.05 2.56 2.67 3.02 ...
##  $ bare.sand    : num  0 0 0 0 0 ...
##  $ fallen.leaves: num  0 1.79 0 0 0 ...
##  $ moss         : num  3.04 1.1 2.4 2.4 0 ...
##  $ herb.layer   : num  4.45 4.56 4.61 4.62 4.62 ...
##  $ reflection   : num  3.91 1.61 3.69 3 2.3 ...
##  $ soil.cat     : Factor w/ 4 levels "dry","damp","wet",..: 3 4 3 3 4 4 4 3 3 3 ...
\end{verbatim}

\begin{Shaded}
\begin{Highlighting}[]
\NormalTok{comm[}\DecValTok{1}\OperatorTok{:}\DecValTok{8}\NormalTok{, }\DecValTok{1}\OperatorTok{:}\DecValTok{8}\NormalTok{]}
\end{Highlighting}
\end{Shaded}

\begin{verbatim}
##   Alopacce Alopcune Alopfabr Arctlute Arctperi Auloalbi Pardlugu Pardmont
## a       25       10        0        0        0        4        0       60
## b        0        2        0        0        0       30        1        1
## c       15       20        2        2        0        9        1       29
## d        2        6        0        1        0       24        1        7
## e        1       20        0        2        0        9        1        2
## f        0        6        0        6        0        6        0       11
## g        2        7        0       12        0       16        1       30
## h        0       11        0        0        0        7       55        2
\end{verbatim}

\hypertarget{basic-functions}{%
\section{Basic Functions}\label{basic-functions}}

\hypertarget{ordination}{%
\subsection{Ordination}\label{ordination}}

We'll first perform a basic principle components analysis of the
community with \texttt{pcwOrd()} and visualize the results with
\texttt{plot\_ord}:

\begin{Shaded}
\begin{Highlighting}[]
\NormalTok{pca =}\StringTok{ }\KeywordTok{pcwOrd}\NormalTok{(comm)}
\KeywordTok{plot_ord}\NormalTok{(pca, }
         \DataTypeTok{main=}\StringTok{'Spiders: Basic PCA'}\NormalTok{)}
\end{Highlighting}
\end{Shaded}

\begin{verbatim}
## Rescaling standard scores by 10.12
\end{verbatim}

\includegraphics{pcwOrd-Vignette_files/figure-latex/pca_pcword-1.pdf}

The \texttt{pca} object is a list of class \texttt{pcwOrd} that contains
a number of items. See the documentation for \texttt{pcwOrd}. Important
ones are: \texttt{\textquotesingle{}Y\_scaled\textquotesingle{}} - the
community matrix after centering, scaling, and/or weighting but before
further analysis;
\texttt{\textquotesingle{}unconstrained\textquotesingle{}} - the
singular value decomosition matrices (left
\texttt{\textquotesingle{}u\textquotesingle{}} and right
\texttt{\textquotesingle{}v\textquotesingle{}}) and values
(\texttt{\textquotesingle{}d\textquotesingle{}}). All relevant
information about this ordination can be calculated from these values.

\hypertarget{eigenvalues}{%
\subsubsection{Eigenvalues}\label{eigenvalues}}

For example, if we want to calculate eigenvalues of this unconstrained
ordination, we need to access the singular values (vector \texttt{d}) of
the unconstrained solution:

\begin{Shaded}
\begin{Highlighting}[]
\NormalTok{pca}\OperatorTok{$}\NormalTok{unconstrained}\OperatorTok{$}\NormalTok{d}\OperatorTok{^}\DecValTok{2}
\end{Highlighting}
\end{Shaded}

\begin{verbatim}
##  [1] 251.7432705  58.7674666  25.3873166  11.7224346   5.6704074   4.1539497
##  [7]   2.7649418   2.3272743   1.6733360   0.8060037   0.6634983   0.1884040
\end{verbatim}

\hypertarget{axis-variances}{%
\subsubsection{Axis Variances}\label{axis-variances}}

We can view these same data by calling ord\_variance, which summarizes
the variance explained by each axis in the solution:

\begin{Shaded}
\begin{Highlighting}[]
\KeywordTok{ord_variance}\NormalTok{(pca)}
\end{Highlighting}
\end{Shaded}

\begin{verbatim}
## $type
## [1] "variance"
## 
## $summary
##              total partialed constrained unconstrained
## value     365.8683         0           0      365.8683
## pct_total 100.0000         0           0      100.0000
## 
## $total
##              Total
## value     365.8683
## pct_group 100.0000
## pct_total 100.0000
## 
## $partialed
##          
## value    
## pct_group
## pct_total
## 
## $constrained
##          
## value    
## pct_group
## pct_total
## 
## $unconstrained
##              uAxis1   uAxis2    uAxis3    uAxis4   uAxis5   uAxis6    uAxis7
## value     251.74327 58.76747 25.387317 11.722435 5.670407 4.153950 2.7649418
## pct_group  68.80707 16.06246  6.938922  3.204004 1.549849 1.135367 0.7557205
## pct_total  68.80707 16.06246  6.938922  3.204004 1.549849 1.135367 0.7557205
##              uAxis8    uAxis9   uAxis10   uAxis11    uAxis12
## value     2.3272743 1.6733360 0.8060037 0.6634983 0.18840395
## pct_group 0.6360962 0.4573602 0.2202989 0.1813489 0.05149502
## pct_total 0.6360962 0.4573602 0.2202989 0.1813489 0.05149502
\end{verbatim}

\hypertarget{scree-plots}{%
\subsubsection{Scree plots}\label{scree-plots}}

We can also visualize variances across axes in a scree plot with
\texttt{ord\_scree()}:

\begin{Shaded}
\begin{Highlighting}[]
\KeywordTok{ord_scree}\NormalTok{(pca)}
\end{Highlighting}
\end{Shaded}

\includegraphics{pcwOrd-Vignette_files/figure-latex/unnamed-chunk-4-1.pdf}

\hypertarget{calculate-scores}{%
\subsubsection{Calculate scores}\label{calculate-scores}}

If you want to make your own biplots, you can calculate row, column,
centeroid, and biplot scores by calling \texttt{ord\_scores}. Scores can
be principle, standard, or contribution - with the latter being raw
singular values (either \texttt{\textquotesingle{}u\textquotesingle{}}
or \texttt{\textquotesingle{}v\textquotesingle{}}). To make plotting
easy, we can also add grouping information from the \texttt{envi} table:

\begin{Shaded}
\begin{Highlighting}[]
\NormalTok{grp =}\StringTok{ }\NormalTok{envi[, }\StringTok{'soil.cat'}\NormalTok{, drop=}\OtherTok{FALSE}\NormalTok{] }\CommentTok{# preserve rownames}
\NormalTok{rowscores =}\StringTok{ }\KeywordTok{ord_scores}\NormalTok{(pca, }
                       \DataTypeTok{choice=}\StringTok{'row'}\NormalTok{, }
                       \DataTypeTok{scaling=}\StringTok{'principle'}\NormalTok{, }
                       \DataTypeTok{axes=}\KeywordTok{c}\NormalTok{(}\DecValTok{1}\OperatorTok{:}\DecValTok{3}\NormalTok{), }
                       \DataTypeTok{add_grouping=}\NormalTok{grp)}
\KeywordTok{head}\NormalTok{(rowscores)}
\end{Highlighting}
\end{Shaded}

\begin{verbatim}
##      Axis1      Axis2      Axis3 soil.cat
## a 1.509623  2.6053593 -0.6811249      wet
## b 1.706576 -0.8793122 -1.0806346  soaking
## c 2.004352  0.8598015 -0.6754797      wet
## d 4.702137 -0.4294234 -1.6207403      wet
## e 6.987565 -1.1952552  3.6085514  soaking
## f 1.641038 -0.4773010 -0.3333297  soaking
\end{verbatim}

When plotting both standard and principle scores, often you'll need to
rescale one of them for visualization:

\begin{Shaded}
\begin{Highlighting}[]
\NormalTok{colscores =}\StringTok{ }\KeywordTok{ord_scores}\NormalTok{(pca, }
                       \DataTypeTok{choice=}\StringTok{'column'}\NormalTok{, }
                       \DataTypeTok{scaling=}\StringTok{'standard'}\NormalTok{, }
                       \DataTypeTok{axes=}\KeywordTok{c}\NormalTok{(}\DecValTok{1}\OperatorTok{:}\DecValTok{3}\NormalTok{))}
\NormalTok{scaled_colscores =}\StringTok{ }\KeywordTok{scale_scores}\NormalTok{(colscores, }
\NormalTok{                                rowscores) }\CommentTok{# can also specify manual scaling factor}
\end{Highlighting}
\end{Shaded}

\begin{verbatim}
## Rescaling standard scores by 8.271
\end{verbatim}

\begin{Shaded}
\begin{Highlighting}[]
\KeywordTok{lapply}\NormalTok{(}\KeywordTok{list}\NormalTok{(}\DataTypeTok{original =}\NormalTok{ colscores, }
            \DataTypeTok{rescaled =}\NormalTok{ scaled_colscores), }
\NormalTok{       head)}
\end{Highlighting}
\end{Shaded}

\begin{verbatim}
## $original
##                Axis1       Axis2       Axis3
## Alopacce -0.03365963  0.16120685  0.06324170
## Alopcune  0.10223352  0.07024215  0.04972208
## Alopfabr -0.04435444 -0.03233246  0.08709767
## Arctlute  0.03001178  0.00254796  0.01295151
## Arctperi -0.01985399 -0.02338508  0.03683295
## Auloalbi  0.10368699 -0.01977533 -0.12316564
## 
## $rescaled
##               Axis1       Axis2      Axis3
## Alopacce -0.2784154  1.33342131  0.5231032
## Alopcune  0.8456238  0.58100745  0.4112758
## Alopfabr -0.3668775 -0.26743774  0.7204277
## Arctlute  0.2482422  0.02107543  0.1071284
## Arctperi -0.1642221 -0.19342952  0.3046635
## Auloalbi  0.8576462 -0.16357151 -1.0187637
\end{verbatim}

\hypertarget{identify-top-features}{%
\subsubsection{Identify top features}\label{identify-top-features}}

Finally, it's useful to identify the top features. The default
\texttt{scaling=\textquotesingle{}contribution\textquotesingle{}} gives
top contribution scores - i.e.~the features that contribute most to the
axes chosen. Here, we identify the spider taxa that have the five
highest mean contributions across the first two axes:

\begin{Shaded}
\begin{Highlighting}[]
\KeywordTok{top_scores}\NormalTok{(pca, }\DataTypeTok{n=}\DecValTok{5}\NormalTok{, }\DataTypeTok{choice=}\StringTok{'column'}\NormalTok{, }\DataTypeTok{scaling=}\StringTok{'contribution'}\NormalTok{, }\DataTypeTok{axes=}\KeywordTok{c}\NormalTok{(}\DecValTok{1}\OperatorTok{:}\DecValTok{2}\NormalTok{))}
\end{Highlighting}
\end{Shaded}

\begin{verbatim}
##          mean_contribution
## Pardmont         0.9681618
## Trocterr         0.6288081
## Pardpull         0.5841149
## Pardnigr         0.4835553
## Alopacce         0.1646834
## attr(,"axes")
## [1] 1 2
\end{verbatim}

\hypertarget{more-ordinations}{%
\section{More ordinations}\label{more-ordinations}}

\hypertarget{log-ratio-analysis}{%
\subsection{Log-ratio analysis}\label{log-ratio-analysis}}

To perform log-ratio analysis:

\begin{enumerate}
\def\labelenumi{\arabic{enumi}.}
\tightlist
\item
  Remove zeros from your data. For simplicity, I'll replace zeros with a
  pseudocount of 0.5. Note that I am \emph{not} adding the pseudocount
  as \texttt{x\ +\ 0.5}, as this will distort the ratios between
  observations.
\item
  Close the community matrix, so that all abundances are relative
  abundances and \texttt{rowSums(x)\ =\ 1}.
\item
  Perform the log-ratio transformation of your choice. The centered
  log-ratio (CLR), where values are centered on the geometric mean, is a
  common choice.
\item
  Run principle components analysis on the log-ratio transformed matrix
\end{enumerate}

\begin{Shaded}
\begin{Highlighting}[]
\CommentTok{# Pseudocounts}
\NormalTok{cc =}\StringTok{ }\NormalTok{comm}
\NormalTok{cc[cc}\OperatorTok{==}\DecValTok{0}\NormalTok{] =}\StringTok{ }\FloatTok{0.5} \CommentTok{# replace zeros without adding to all values!}

\CommentTok{# Close the community}
\NormalTok{closed_comm =}\StringTok{ }\KeywordTok{sweep}\NormalTok{(cc, }\DecValTok{1}\NormalTok{, }\KeywordTok{rowSums}\NormalTok{(comm), }\StringTok{'/'}\NormalTok{) }\CommentTok{# close to relative abundance}

\CommentTok{# perform centered log-ratio}
\NormalTok{geo_mean =}\StringTok{ }\KeywordTok{apply}\NormalTok{(closed_comm, }\DecValTok{1}\NormalTok{, }
                 \ControlFlowTok{function}\NormalTok{(x) \{}
                   \KeywordTok{exp}\NormalTok{(}\KeywordTok{sum}\NormalTok{(}\KeywordTok{log}\NormalTok{(x))}\OperatorTok{/}\KeywordTok{length}\NormalTok{(x))}
\NormalTok{                 \})}
\NormalTok{lr_comm =}\StringTok{  }\KeywordTok{sweep}\NormalTok{(closed_comm, }\DecValTok{1}\NormalTok{, geo_mean, }\StringTok{'/'}\NormalTok{)}
\NormalTok{lr_comm =}\StringTok{ }\KeywordTok{log}\NormalTok{(lr_comm, }\DecValTok{2}\NormalTok{)}

\NormalTok{lra =}\StringTok{ }\KeywordTok{pcwOrd}\NormalTok{(lr_comm)}
\KeywordTok{plot_ord}\NormalTok{(lra, }
         \DataTypeTok{main=}\StringTok{'Spiders: Log Ratio Analysis'}\NormalTok{)}
\end{Highlighting}
\end{Shaded}

\begin{verbatim}
## Rescaling standard scores by 0.7505
\end{verbatim}

\includegraphics{pcwOrd-Vignette_files/figure-latex/unnamed-chunk-8-1.pdf}

\hypertarget{weighted-ordinations}{%
\subsection{Weighted ordinations}\label{weighted-ordinations}}

To weight either rows or columns, for example by sequencing depth or
column prevalence, you have three options:

\begin{enumerate}
\def\labelenumi{\arabic{enumi}.}
\tightlist
\item
  Tell pcwOrd to calculate weights automatically by setting
  \texttt{weight\_rows=TRUE} or \texttt{weight\_columns=TRUE}. This uses
  row or column masses as weights (via \texttt{rowSums}, for example).
\item
  Specify a vector of length \texttt{nrow()} or \texttt{ncol()} with the
  weights.
\item
  Input the result of \texttt{easyCODA::CLR()}, which returns a weighted
  CLR transformation and row-weights. This approach is demonstrated
  below:
\end{enumerate}

Here is a weighted LRA:

\begin{Shaded}
\begin{Highlighting}[]
\NormalTok{wlr_comm =}\StringTok{ }\KeywordTok{CLR}\NormalTok{(closed_comm) }\CommentTok{# CLR based on column-weighted geometric mean, plus column weights}

\NormalTok{w_lra =}\StringTok{ }\KeywordTok{pcwOrd}\NormalTok{(wlr_comm)}
\KeywordTok{plot_ord}\NormalTok{(w_lra, }
         \DataTypeTok{main=}\StringTok{'Spiders: Column-Weighted LRA'}\NormalTok{)}
\end{Highlighting}
\end{Shaded}

\begin{verbatim}
## Rescaling standard scores by 0.1744
\end{verbatim}

\includegraphics{pcwOrd-Vignette_files/figure-latex/unnamed-chunk-9-1.pdf}

In this situation, the results of weighted and unweighted ordinations
are approximately the same. See Greenacre and Lewi (2009) for a
discussion of when weighted log-ratio analysis might be advantageous
over a non-weighted log ratio analysis.

\hypertarget{constrained-ordination}{%
\subsection{Constrained ordination}\label{constrained-ordination}}

We might be interested in how these spiders associate with environmental
variables. For example, we might be interested in whether moss and
reflection structures spider communities. A constrained ordination like
redundancy analysis will do this: pcwOrd will regress the community
matrix against moss and reflectivity, and then ordinate the fitted
values.

\begin{Shaded}
\begin{Highlighting}[]
\NormalTok{X =}\StringTok{ }\NormalTok{envi[, }\KeywordTok{c}\NormalTok{(}\StringTok{'moss'}\NormalTok{, }\StringTok{'reflection'}\NormalTok{)] }\CommentTok{# preserve rownames}
\NormalTok{rda =}\StringTok{ }\KeywordTok{pcwOrd}\NormalTok{(comm, }\DataTypeTok{X=}\NormalTok{X)}

\KeywordTok{plot_ord}\NormalTok{(rda, }
         \DataTypeTok{main=}\StringTok{'Spiders by Moss Coverage'}\NormalTok{, }
         \DataTypeTok{color_legend_position=}\KeywordTok{c}\NormalTok{(}\FloatTok{0.8}\NormalTok{, }\DecValTok{0}\NormalTok{))}
\end{Highlighting}
\end{Shaded}

\begin{verbatim}
## Rescaling standard scores by 8.774
\end{verbatim}

\begin{verbatim}
## Rescaling standard scores by 9.116
\end{verbatim}

\includegraphics{pcwOrd-Vignette_files/figure-latex/unnamed-chunk-10-1.pdf}

In this plot, colors show moss coverage, the first column in X.

\hypertarget{significance-testing}{%
\subsubsection{Significance testing}\label{significance-testing}}

To test whether spiders vary with wetness class, run a PERMANOVA on the
ordination with \texttt{permute\_ord()}:

\begin{Shaded}
\begin{Highlighting}[]
\NormalTok{pp =}\StringTok{ }\KeywordTok{permute_ord}\NormalTok{(rda)}
\NormalTok{pp}
\end{Highlighting}
\end{Shaded}

\begin{verbatim}
##   permute_on total_variance variance_after_partialing   fitted residuals num_df
## 1    partial       365.8683                  365.8683 98.72652  267.1418      2
##   denom_df   F_stat p_val F_perm
## 1       25 4.619575 0.001    999
\end{verbatim}

\texttt{permute\_ord()} has a number of permutation models that parallel
the options in \texttt{vegan::permutest()} - and \texttt{permute\_ord()}
will return the same results as \texttt{vegan::permutest()} with the
same randomization seed. We see a p-value of 0.001 after 999
permutations.

\hypertarget{partial-ordination}{%
\subsection{Partial ordination}\label{partial-ordination}}

Say we want to look at how moass affects spider communities
independently of moisture class. We can partial this out as Z:

\begin{Shaded}
\begin{Highlighting}[]
\NormalTok{Z =}\StringTok{ }\NormalTok{envi[, }\StringTok{'soil.cat'}\NormalTok{, drop=}\OtherTok{FALSE}\NormalTok{] }\CommentTok{# preserve rownames}
\NormalTok{X =}\StringTok{ }\NormalTok{envi[, }\KeywordTok{c}\NormalTok{(}\StringTok{'moss'}\NormalTok{, }\StringTok{'reflection'}\NormalTok{), drop=}\OtherTok{FALSE}\NormalTok{]}

\NormalTok{pcrda =}\StringTok{ }\KeywordTok{pcwOrd}\NormalTok{(closed_comm, X, Z)}

\KeywordTok{plot_ord}\NormalTok{(pcrda, }
         \DataTypeTok{main=}\StringTok{'Spiders by moss and reflection, partialed by wetness'}\NormalTok{, }
         \DataTypeTok{shape_legend_position=}\KeywordTok{c}\NormalTok{(}\OperatorTok{-}\FloatTok{0.04}\NormalTok{, }\FloatTok{0.01}\NormalTok{))}
\end{Highlighting}
\end{Shaded}

\begin{verbatim}
## Rescaling standard scores by 0.03821
\end{verbatim}

\begin{verbatim}
## Rescaling standard scores by 0.05596
\end{verbatim}

\includegraphics{pcwOrd-Vignette_files/figure-latex/unnamed-chunk-12-1.pdf}

Each of the wetness categories is centered at zero, as expected due to
partialing. Moss increases to the bottom left (darker red, and arrows).
This model is highly significant:

\begin{Shaded}
\begin{Highlighting}[]
\KeywordTok{permute_ord}\NormalTok{(pcrda)}
\end{Highlighting}
\end{Shaded}

\begin{verbatim}
##   permute_on total_variance variance_after_partialing      fitted   residuals
## 1    partial     0.01927485                0.01070139 0.004584283 0.006117103
##   num_df denom_df   F_stat p_val F_perm
## 1      2       22 8.243627 0.001    999
\end{verbatim}

\hypertarget{partialed-constrained-weighted-ordination}{%
\subsection{Partialed, constrained, weighted
ordination}\label{partialed-constrained-weighted-ordination}}

The reason for this package is to perform partial, constrained, weighted
ordinations. Here is a weighted log-ratio analysis of the spider
community, constrained by moss and reflection, with moisture category
partialed out:

\begin{Shaded}
\begin{Highlighting}[]
\NormalTok{Y =}\StringTok{ }\KeywordTok{CLR}\NormalTok{(closed_comm)}

\NormalTok{pcw_ord =}\StringTok{ }\KeywordTok{pcwOrd}\NormalTok{(Y, X, Z)}

\KeywordTok{plot_ord}\NormalTok{(pcw_ord, }
         \DataTypeTok{main=}\StringTok{'Partialed, Constrained, Weighted Log-Ratio Analysis'}\NormalTok{, }
         \DataTypeTok{shape_legend_position=}\KeywordTok{c}\NormalTok{(}\OperatorTok{-}\FloatTok{0.4}\NormalTok{, }\FloatTok{0.1}\NormalTok{))}
\end{Highlighting}
\end{Shaded}

\begin{verbatim}
## Rescaling standard scores by 0.1324
\end{verbatim}

\begin{verbatim}
## Rescaling standard scores by 0.6076
\end{verbatim}

\includegraphics{pcwOrd-Vignette_files/figure-latex/unnamed-chunk-14-1.pdf}

\hypertarget{comparison-to-vegan-and-easycoda}{%
\section{\texorpdfstring{Comparison to \texttt{vegan} and
\texttt{easyCODA}}{Comparison to vegan and easyCODA}}\label{comparison-to-vegan-and-easycoda}}

\texttt{pcwOrd} gives the same results as comparable \texttt{vegan} and
\texttt{easycCODA} functions.

Returning to eigenvalues: these eigenvalues are exactly the same as
calculated by easyCODA's PCA, and solutions are the same:

\begin{Shaded}
\begin{Highlighting}[]
\NormalTok{easy_pca =}\StringTok{ }\KeywordTok{PCA}\NormalTok{(comm, }\DataTypeTok{weight=}\OtherTok{FALSE}\NormalTok{)}

\KeywordTok{rbind}\NormalTok{(}\DataTypeTok{pcwOrd =}\NormalTok{ pca}\OperatorTok{$}\NormalTok{unconstrained}\OperatorTok{$}\NormalTok{d}\OperatorTok{^}\DecValTok{2}\NormalTok{,}
      \DataTypeTok{easyCODA =}\NormalTok{ easy_pca}\OperatorTok{$}\NormalTok{sv}\OperatorTok{^}\DecValTok{2}
\NormalTok{)}
\end{Highlighting}
\end{Shaded}

\begin{verbatim}
##              [,1]     [,2]     [,3]     [,4]     [,5]    [,6]     [,7]     [,8]
## pcwOrd   251.7433 58.76747 25.38732 11.72243 5.670407 4.15395 2.764942 2.327274
## easyCODA 251.7433 58.76747 25.38732 11.72243 5.670407 4.15395 2.764942 2.327274
##              [,9]     [,10]     [,11]    [,12]
## pcwOrd   1.673336 0.8060037 0.6634983 0.188404
## easyCODA 1.673336 0.8060037 0.6634983 0.188404
\end{verbatim}

Eigenvalues are not identical to \texttt{vegan} eigenvalues, because
\texttt{pcwOrd} (and \texttt{easyCODA}) weights the initial community
matrix rows by \texttt{1/sqrt(nrow)}, while \texttt{vegan} weights rows
by \texttt{1/sqrt(nrow-1)}. You can access \texttt{vegan}-style
weighting, which will agree with vegan results:

\begin{Shaded}
\begin{Highlighting}[]
\NormalTok{veganlike_pca =}\StringTok{ }\KeywordTok{pcwOrd}\NormalTok{(comm, }\DataTypeTok{as_vegan=}\OtherTok{TRUE}\NormalTok{)}
\NormalTok{vegan_pca =}\StringTok{ }\KeywordTok{rda}\NormalTok{(comm)}

\KeywordTok{rbind}\NormalTok{(}\DataTypeTok{pcwOrd =}\NormalTok{ veganlike_pca}\OperatorTok{$}\NormalTok{unconstrained}\OperatorTok{$}\NormalTok{d}\OperatorTok{^}\DecValTok{2}\NormalTok{, }
      \DataTypeTok{vegan =}\NormalTok{ vegan_pca}\OperatorTok{$}\NormalTok{CA}\OperatorTok{$}\NormalTok{eig)}
\end{Highlighting}
\end{Shaded}

\begin{verbatim}
##             PC1      PC2      PC3      PC4      PC5     PC6      PC7      PC8
## pcwOrd 3132.805 731.3285 315.9311 145.8792 70.56507 51.6936 34.40816 28.96164
## vegan  3132.805 731.3285 315.9311 145.8792 70.56507 51.6936 34.40816 28.96164
##             PC9     PC10     PC11     PC12
## pcwOrd 20.82374 10.03027 8.256867 2.344583
## vegan  20.82374 10.03027 8.256867 2.344583
\end{verbatim}

These weightings don't change relative variance or relative distances,
only absolute variances and distances. Generally, we're concerned with
relative variances and distances, so the choice of (equal) weighting
doesn't matter.

\begin{Shaded}
\begin{Highlighting}[]
\KeywordTok{list}\NormalTok{(}
  \DataTypeTok{coda_weightings =} \KeywordTok{ord_variance}\NormalTok{(pca)}\OperatorTok{$}\NormalTok{unconstrained,}
  \DataTypeTok{vegan_weightings =} \KeywordTok{ord_variance}\NormalTok{(veganlike_pca)}\OperatorTok{$}\NormalTok{unconstrained}
\NormalTok{)}
\end{Highlighting}
\end{Shaded}

\begin{verbatim}
## $coda_weightings
##              uAxis1   uAxis2    uAxis3    uAxis4   uAxis5   uAxis6    uAxis7
## value     251.74327 58.76747 25.387317 11.722435 5.670407 4.153950 2.7649418
## pct_group  68.80707 16.06246  6.938922  3.204004 1.549849 1.135367 0.7557205
## pct_total  68.80707 16.06246  6.938922  3.204004 1.549849 1.135367 0.7557205
##              uAxis8    uAxis9   uAxis10   uAxis11    uAxis12
## value     2.3272743 1.6733360 0.8060037 0.6634983 0.18840395
## pct_group 0.6360962 0.4573602 0.2202989 0.1813489 0.05149502
## pct_total 0.6360962 0.4573602 0.2202989 0.1813489 0.05149502
## 
## $vegan_weightings
##               uAxis1    uAxis2     uAxis3     uAxis4    uAxis5    uAxis6
## value     3132.80514 731.32847 315.931051 145.879186 70.565070 51.693597
## pct_group   68.80707  16.06246   6.938922   3.204004  1.549849  1.135367
## pct_total   68.80707  16.06246   6.938922   3.204004  1.549849  1.135367
##               uAxis7     uAxis8     uAxis9    uAxis10   uAxis11    uAxis12
## value     34.4081650 28.9616352 20.8237366 10.0302687 8.2568673 2.34458251
## pct_group  0.7557205  0.6360962  0.4573602  0.2202989 0.1813489 0.05149502
## pct_total  0.7557205  0.6360962  0.4573602  0.2202989 0.1813489 0.05149502
\end{verbatim}

\texttt{vegan::permutest()} gives the same as
\texttt{pcwOrd::permute\_ord()}, if you set the same seed.

\begin{Shaded}
\begin{Highlighting}[]
\NormalTok{vegan_rda =}\StringTok{ }\KeywordTok{rda}\NormalTok{(comm, X)}
\NormalTok{veganlike_rda =}\StringTok{ }\KeywordTok{pcwOrd}\NormalTok{(comm, X, }\DataTypeTok{as_vegan=}\OtherTok{TRUE}\NormalTok{)}

\KeywordTok{set.seed}\NormalTok{(}\DecValTok{445}\NormalTok{)}
\NormalTok{vegan_permutes =}\StringTok{ }\KeywordTok{permutest}\NormalTok{(vegan_rda, }
                           \DataTypeTok{permutations=}\DecValTok{999}\NormalTok{)}
\KeywordTok{set.seed}\NormalTok{(}\DecValTok{445}\NormalTok{)}
\NormalTok{pcw_permutes =}\StringTok{ }\KeywordTok{permute_ord}\NormalTok{(veganlike_rda)}
\NormalTok{keep =}\StringTok{ }\KeywordTok{c}\NormalTok{(}\StringTok{'fitted'}\NormalTok{, }\StringTok{'residuals'}\NormalTok{, }\StringTok{'num_df'}\NormalTok{, }\StringTok{'denom_df'}\NormalTok{, }\StringTok{'F_stat'}\NormalTok{, }\StringTok{'p_val'}\NormalTok{)}

\KeywordTok{list}\NormalTok{(}\DataTypeTok{vegan =}\NormalTok{ vegan_permutes,}
     \DataTypeTok{pcwOrd =}\NormalTok{ pcw_permutes[keep])}
\end{Highlighting}
\end{Shaded}

\begin{verbatim}
## $vegan
## 
## Permutation test for rda under reduced model 
## 
## Permutation: free
## Number of permutations: 999
##  
## Model: rda(X = comm, Y = X)
## Permutation test for all constrained eigenvalues
##          Df Inertia      F Pr(>F)   
## Model     2  1228.6 4.6196  0.004 **
## Residual 25  3324.4                 
## ---
## Signif. codes:  0 '***' 0.001 '**' 0.01 '*' 0.05 '.' 0.1 ' ' 1
## 
## $pcwOrd
##     fitted residuals num_df denom_df   F_stat p_val
## 1 1228.597  3324.431      2       25 4.619575 0.004
\end{verbatim}

\hypertarget{additional-plotting-options}{%
\section{Additional plotting
options}\label{additional-plotting-options}}

\hypertarget{added-categories}{%
\subsection{Added categories}\label{added-categories}}

Add groupings to any ordination. They can be continuous.

\begin{Shaded}
\begin{Highlighting}[]
\NormalTok{row_group =}\StringTok{ }\NormalTok{envi[, }\StringTok{'soil.cat'}\NormalTok{, drop=}\OtherTok{FALSE}\NormalTok{]}
\KeywordTok{plot_ord}\NormalTok{(pca, }
         \DataTypeTok{row_group=}\NormalTok{row_group, }
         \DataTypeTok{main=}\StringTok{'PCA with groupings by soil moisture class'}\NormalTok{)}
\end{Highlighting}
\end{Shaded}

\begin{verbatim}
## Rescaling standard scores by 10.12
\end{verbatim}

\includegraphics{pcwOrd-Vignette_files/figure-latex/unnamed-chunk-19-1.pdf}

\hypertarget{different-axes}{%
\subsection{Different axes}\label{different-axes}}

Plot different axes

\begin{Shaded}
\begin{Highlighting}[]
\KeywordTok{plot_ord}\NormalTok{(pca, }
         \DataTypeTok{row_group=}\NormalTok{row_group, }
         \DataTypeTok{axes=}\KeywordTok{c}\NormalTok{(}\DecValTok{3}\NormalTok{,}\DecValTok{5}\NormalTok{), }
         \DataTypeTok{main=}\StringTok{'PCA axes 3 and 5'}\NormalTok{,}
         \DataTypeTok{color_legend_position=}\KeywordTok{c}\NormalTok{(}\FloatTok{0.6}\NormalTok{, }\DecValTok{0}\NormalTok{))}
\end{Highlighting}
\end{Shaded}

\begin{verbatim}
## Rescaling standard scores by 3.592
\end{verbatim}

\includegraphics{pcwOrd-Vignette_files/figure-latex/unnamed-chunk-20-1.pdf}

\hypertarget{limit-column-labels}{%
\subsection{Limit column labels}\label{limit-column-labels}}

Reduce clutter by only labeling the top columns (by plotted score)

\begin{Shaded}
\begin{Highlighting}[]
\KeywordTok{plot_ord}\NormalTok{(pca, }
         \DataTypeTok{row_group=}\NormalTok{row_group, }
         \DataTypeTok{max_labels=}\DecValTok{5}\NormalTok{)}
\end{Highlighting}
\end{Shaded}

\begin{verbatim}
## Rescaling standard scores by 10.12
\end{verbatim}

\includegraphics{pcwOrd-Vignette_files/figure-latex/unnamed-chunk-21-1.pdf}

\hypertarget{colors-and-shapes}{%
\subsection{colors and shapes}\label{colors-and-shapes}}

Change the colors of text, groupings, arrows, and change shapes.

\begin{Shaded}
\begin{Highlighting}[]
\NormalTok{new_colors =}\StringTok{ }\KeywordTok{hcl.colors}\NormalTok{(}\DecValTok{4}\NormalTok{, }\DataTypeTok{palette=}\StringTok{'Viridis'}\NormalTok{)}

\KeywordTok{plot_ord}\NormalTok{(pca, }
         \DataTypeTok{row_group=}\NormalTok{row_group, }
         \DataTypeTok{max_labels=}\DecValTok{5}\NormalTok{, }
         \DataTypeTok{discrete_scale=}\NormalTok{new_colors,}
         \DataTypeTok{col_text=}\StringTok{'green'}\NormalTok{, }
         \DataTypeTok{main=}\StringTok{'Fluorescent PCA'}\NormalTok{)}
\end{Highlighting}
\end{Shaded}

\begin{verbatim}
## Rescaling standard scores by 10.12
\end{verbatim}

\includegraphics{pcwOrd-Vignette_files/figure-latex/unnamed-chunk-22-1.pdf}

If you constrain an ordination with categories, the plot function will
automatically recognize these and plot them as named centeroids:

\begin{Shaded}
\begin{Highlighting}[]
\NormalTok{X =}\StringTok{ }\NormalTok{envi[, }\StringTok{'soil.cat'}\NormalTok{, drop=}\OtherTok{FALSE}\NormalTok{]}
\NormalTok{Y =}\StringTok{ }\KeywordTok{CLR}\NormalTok{(closed_comm)}

\NormalTok{cat_cwLRA =}\StringTok{ }\KeywordTok{pcwOrd}\NormalTok{(Y, X)}

\KeywordTok{plot_ord}\NormalTok{(cat_cwLRA, }
         \DataTypeTok{main=}\StringTok{'Spiders wLRA with Centeroids'}\NormalTok{)}
\end{Highlighting}
\end{Shaded}

\begin{verbatim}
## Rescaling standard scores by 0.1624
\end{verbatim}

\includegraphics{pcwOrd-Vignette_files/figure-latex/unnamed-chunk-23-1.pdf}

\hypertarget{screeplots-as-grobs}{%
\subsection{Screeplots as grobs}\label{screeplots-as-grobs}}

Screeplots can be produced as grobs (via ggplot2) for downstream
manipulation, saving as R objects, and arranging with
\texttt{gridExtra}. If you want to do this with ordinations, you're on
your own!

\begin{Shaded}
\begin{Highlighting}[]
\KeywordTok{ord_scree}\NormalTok{(cat_cwLRA, }\DataTypeTok{main=}\StringTok{'ggplot2 Scree Plot'}\NormalTok{, }\DataTypeTok{as_grob=}\OtherTok{TRUE}\NormalTok{) }\OperatorTok{+}\StringTok{ }
\StringTok{  }\KeywordTok{theme_dark}\NormalTok{()}
\end{Highlighting}
\end{Shaded}

\begin{verbatim}
## Loading required package: ggplot2
\end{verbatim}

\includegraphics{pcwOrd-Vignette_files/figure-latex/unnamed-chunk-24-1.pdf}

\end{document}
